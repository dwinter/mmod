\nonstopmode{}
\documentclass[letterpaper]{book}
\usepackage[times,hyper]{Rd}
\usepackage{makeidx}
\usepackage[utf8,latin1]{inputenc}
\makeindex{}
\begin{document}
\chapter*{}
\begin{center}
{\textbf{\huge Package `mmod'}}
\par\bigskip{\large \today}
\end{center}
\begin{description}
\raggedright{}
\item[Version]\AsIs{0.1}
\item[Date]\AsIs{2011-11-22}
\item[Title]\AsIs{Modern measures of population divergence}
\item[Author]\AsIs{David Winter}
\item[Maintainer]\AsIs{David Winter }\email{david.winter@gmail.com}\AsIs{}
\item[Depends]\AsIs{R (>= 2.6.0), adegenet}
\item[ZipData]\AsIs{no}
\item[Description]\AsIs{mmod provides functions}
\item[License]\AsIs{MIT}
\item[URL]\AsIs{https://github.com/dwinter [update!]}
\item[Collate]\AsIs{'diff\_stats.R' 'D\_Jost.R' 'Gst\_Hedrick.R' 'Gst\_Nei.R'
'harmonic\_mean.R' 'jacknife\_pop.R' 'pairwise\_D.R'
'pairwise\_Gst\_Hedrick.R' 'pairwise\_Gst\_Nei.R'}
\end{description}
\Rdcontents{\R{} topics documented:}
\inputencoding{utf8}
\HeaderA{diff\_stats}{Calculate differentiation statistics for a genind objects}{diff.Rul.stats}
%
\begin{Description}\relax
This function calculates three different statistics of
differentiaion for a genetic dataset. Nei's Gst,
Hedrick's G'st and Jost's D
\end{Description}
%
\begin{Usage}
\begin{verbatim}
  diff_stats(x)
\end{verbatim}
\end{Usage}
%
\begin{Arguments}
\begin{ldescription}
\item[\code{x}] genind object (from package adegenet)
\end{ldescription}
\end{Arguments}
%
\begin{References}\relax
Jost, L. (2008), GST and its relatives do not measure
differentiation. Molecular Ecology, 17: 4015–4026
\end{References}
%
\begin{Examples}
\begin{ExampleCode}
data(nancycats)
diff_stats(nancycats)
\end{ExampleCode}
\end{Examples}
\inputencoding{utf8}
\HeaderA{D\_Jost}{Calculate Jost's D using}{D.Rul.Jost}
%
\begin{Description}\relax
This function calculates Jost's D
\end{Description}
%
\begin{Usage}
\begin{verbatim}
  D_Jost(x)
\end{verbatim}
\end{Usage}
%
\begin{Arguments}
\begin{ldescription}
\item[\code{x}] genind object (from package adegenet)
\end{ldescription}
\end{Arguments}
%
\begin{Examples}
\begin{ExampleCode}
data(nancycats)
D_Jost(nancycats)
\end{ExampleCode}
\end{Examples}
\inputencoding{utf8}
\HeaderA{Gst\_Hedrick}{Calculate Nei's Gst using estimators for Hs and Ht}{Gst.Rul.Hedrick}
%
\begin{Description}\relax
This function calculates G'st, Hedrick's correction to
Gst accounting for observed Hs. Nei and Chesser's
estimators of Hs and Ht are used
\end{Description}
%
\begin{Usage}
\begin{verbatim}
  Gst_Hedrick(x)
\end{verbatim}
\end{Usage}
%
\begin{Arguments}
\begin{ldescription}
\item[\code{x}] genind object (from package adegenet)
\end{ldescription}
\end{Arguments}
%
\begin{Examples}
\begin{ExampleCode}
data(nancycats)
Gst_Hedrick(nancycats)
\end{ExampleCode}
\end{Examples}
\inputencoding{utf8}
\HeaderA{Gst\_Nei}{Calculate Nei's Gst using estimators for Hs and Ht}{Gst.Rul.Nei}
%
\begin{Description}\relax
This function calculates Gst following Nei's method and
using Nei and Chesser's estimators for Hs and Ht
\end{Description}
%
\begin{Usage}
\begin{verbatim}
  Gst_Nei(x)
\end{verbatim}
\end{Usage}
%
\begin{Arguments}
\begin{ldescription}
\item[\code{x}] genind object (from package adegenet)
\end{ldescription}
\end{Arguments}
%
\begin{Examples}
\begin{ExampleCode}
data(nancycats)
Gst_Nei(nancycats)
\end{ExampleCode}
\end{Examples}
\inputencoding{utf8}
\HeaderA{harmonic\_mean}{Harmonic mean}{harmonic.Rul.mean}
%
\begin{Description}\relax
Calculate the harmonic mean of a numeric vector
\end{Description}
%
\begin{Usage}
\begin{verbatim}
  harmonic_mean(x)
\end{verbatim}
\end{Usage}
%
\begin{Arguments}
\begin{ldescription}
\item[\code{x}] numeric vector
\end{ldescription}
\end{Arguments}
%
\begin{Examples}
\begin{ExampleCode}
data(nancycats)
pop.sizes <- table(pop(nancycats))
harmonic_mean(pop.sizes)
\end{ExampleCode}
\end{Examples}
\inputencoding{utf8}
\HeaderA{jacknife\_populations}{Calculate differentiation stats for a jacknife sample of a Genind opject}{jacknife.Rul.populations}
%
\begin{Description}\relax
Calcutes
\end{Description}
%
\begin{Usage}
\begin{verbatim}
  jacknife_populations(x, sample_frac = 0.5, nreps = 1000)
\end{verbatim}
\end{Usage}
%
\begin{Arguments}
\begin{ldescription}
\item[\code{x}] genind object (from package adegenet)

\item[\code{sample\_frac}] fraction of pops to sample in each
replication (default 0.5)

\item[\code{nreps}] number of jacknife replicates to run
(default 1000)
\end{ldescription}
\end{Arguments}
%
\begin{Examples}
\begin{ExampleCode}
## Not run: 
data(nancycats)
jacknife_populations(nancycats)

## End(Not run)
\end{ExampleCode}
\end{Examples}
\inputencoding{utf8}
\HeaderA{pairwise\_D}{Calculates pairwise values of Jost's D}{pairwise.Rul.D}
%
\begin{Description}\relax
This function calculates Jost's D, a measure of genetic
differentiation, between all combinations of populaitons
in a genind object.
\end{Description}
%
\begin{Usage}
\begin{verbatim}
  pairwise_D(x)
\end{verbatim}
\end{Usage}
%
\begin{Arguments}
\begin{ldescription}
\item[\code{x}] genind object (from package adegenet)
\end{ldescription}
\end{Arguments}
%
\begin{Examples}
\begin{ExampleCode}
data(nancycats)
pairwise_D(nancycats[1:26,])
\end{ExampleCode}
\end{Examples}
\inputencoding{utf8}
\HeaderA{pairwise\_Gst\_Hedrick}{Calculates pairwise values of Hedrick's G'st}{pairwise.Rul.Gst.Rul.Hedrick}
%
\begin{Description}\relax
This function calculates Hedrick's G'st, a measure of
genetic differentiation, between all combinations of
populaitons in a genind object.
\end{Description}
%
\begin{Usage}
\begin{verbatim}
  pairwise_Gst_Hedrick(x)
\end{verbatim}
\end{Usage}
%
\begin{Arguments}
\begin{ldescription}
\item[\code{x}] genind object (from package adegenet)
\end{ldescription}
\end{Arguments}
%
\begin{Examples}
\begin{ExampleCode}
data(nancycats)
pairwise_Gst_Hedrick(nancycats[1:26,])
\end{ExampleCode}
\end{Examples}
\inputencoding{utf8}
\HeaderA{pairwise\_Gst\_Nei}{Calculates pairwise values of Nei's Gst}{pairwise.Rul.Gst.Rul.Nei}
%
\begin{Description}\relax
This function calculates Nei's Gst, a measure of genetic
differentiation, between all combinations of populaitons
in a genind object.
\end{Description}
%
\begin{Usage}
\begin{verbatim}
  pairwise_Gst_Nei(x)
\end{verbatim}
\end{Usage}
%
\begin{Arguments}
\begin{ldescription}
\item[\code{x}] genind object (from package adegenet)
\end{ldescription}
\end{Arguments}
%
\begin{Examples}
\begin{ExampleCode}
data(nancycats)
pairwise_Gst_Nei(nancycats[1:26,])
\end{ExampleCode}
\end{Examples}
\printindex{}
\end{document}
